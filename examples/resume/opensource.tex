%-------------------------------------------------------------------------------
%	SECTION TITLE
%-------------------------------------------------------------------------------
\cvsection{Open Source Projects}


%-------------------------------------------------------------------------------
%	CONTENT
%-------------------------------------------------------------------------------
\begin{cventries}

    \cventry
    {} % Job title
    {\href{https://github.com/volcano-sh/volcano}{volcano.io}} % Organization
    {GitHub} % Location
    {Apr. 2023} % Date(s)
    {
      \begin{cvitems} % Description(s) of tasks/responsibilities
        \item Expanded functionalities for job scheduling in the Volcano project, integrated Volcano with a machine learning platform, and provided support for the machine learning platform.
        \item Bug fixes related to preemption functionality in the Volcano open source project.
        \item Utilizing the Volcano plugin mechanism for container development, inference, and other tasks, developed plugins such as sshpiper and ingress.
      \end{cvitems}
    }

    \cventry
    {} % Job title
    {\href{https://github.com/karmada-io/karmada}{karmada.io}} % Organization
    {GitHub} % Location
    {Dec. 2023} % Date(s)
    {
      \begin{cvitems} % Description(s) of tasks/responsibilities
        \item Development and refactoring of certain features in Karmada, integration of a heterogeneous fusion system with Karmada, providing support for multi-cluster management and scheduling.
        \item Advancing multi-cluster scheduling support for the Karmada community, promoting support for machine learning platforms within the multi-cluster scheduling framework, and driving the development of batch scheduling systems for multi-clusters, including big data and machine learning workloads.
        \item Built upon the Karmada multi-cluster scheduling framework, developed a proprietary multi-cluster heterogeneous fusion scheduling system (containers, Supercomputing Center, virtual machines). Implemented the system's deployment on the China Computing NET scheduling platform.
      \end{cvitems}
    }

    \cventry
    {} % Job title
    {\href{https://github.com/kubeedge/sedna}{kubeedge.io}} % Organization
    {GitHub} % Location
    {Jul. 2023} % Date(s)
    {
      \begin{cvitems} % Description(s) of tasks/responsibilities
        \item Contributed to the Sedna open-source project, introducing new features such as synchronization configuration and pull request templates.
      \end{cvitems}
    }

    \cventry
    {} % Job title
    {\href{https://github.com/kubeedge/kubeedge}{kubeedge.io}} % Organization
    {GitHub} % Location
    {Feb. 2022} % Date(s)
    {
      \begin{cvitems} % Description(s) of tasks/responsibilities
        \item Contributed to the KubeEdge open-source project by expanding Helm configurations and providing more convenient configuration options.
      \end{cvitems}
    }

  \cventry
    {} % Job title
    {\href{https://github.com/istio/istio.io}{istio.io}} % Organization
    {GitHub} % Location
    {May. 2022} % Date(s)
    {
      \begin{cvitems} % Description(s) of tasks/responsibilities
        \item Maintenance of Istio official documentation in Chinese, including the translation of Istio release 1.1 update documents.
      \end{cvitems}
    }

  \cventry
    {} % Job title
    {\href{https://github.com/haiker2011/awesome-nlp-sentiment-analysis}{awesome-nlp-sentiment-analysis}} % Organization
    {GitHub} % Location
    {Jan. 2024} % Date(s)
    {
      \begin{cvitems} % Description(s) of tasks/responsibilities
        \item Collecting datasets, papers, and open-source implementations in the field of Natural Language Processing (NLP), with a particular focus on sentiment analysis, emotion cause recognition, and the extraction of evaluation targets and terms, 576 stars
      \end{cvitems}
    }

  \cventry
    {} % Job title
    {\href{https://github.com/servicemesher/getting-started-with-knative}{getting-started-with-knative}} % Organization
    {GitHub} % Location
    {Feb. 2019} % Date(s)
    {
      \begin{cvitems} % Description(s) of tasks/responsibilities
        \item Translation of Chapter 3 of "Getting Started with Knative (Chinese Version)" with reviews for other chapters, 224 stars, To be printed and published by Pivotal Corporation
      \end{cvitems}
    }

\end{cventries}